\section{QR Decomposition by Gram-Schmidt Orthogonalization}

\begin{bigidea}
The QR decomposition is given by $A = QR$ where $Q$ is an orthogonal matrix and $R$ is an upper triangular matrix. There are several ways to compute the QR decomposition including Gram-Schmidt orthogonalization and elementary reflectors.
\end{bigidea}

\begin{definition}
A matrix $A$ is {\bf orthogonal} \cite[p.424]{KN} if $A^TA = AA^T = I$.
\end{definition}

\begin{proposition}
 If $A$ is an orthogonal matrix, then:
\begin{itemize}
\item $\| A \bs{x} \| = \| \bs{x} \|$ for all $\bs{x} \in \mathbb{R}^n$ since
$$
\| A \bs{x} \|^2 = (A \bs{x})^T A \bs{x} = \bs{x}^T A^T A \bs{x} = \bs{x}^T \bs{x} = \| \bs{x} \|^2
$$
\item the columns of $A$ are orthonormal.
\item the rows of $A$ are orthonormal.
\end{itemize}
See \cite[p.424]{KN}.
\end{proposition}

\begin{example}
Rotations and reflections are examples of orthogonal matrices.
\end{example}

\begin{note}
An orthogonal matrix and an orthogonal projector are {\bf not} the same thing but they are related. If $P$ is an orthogonal projector then $Q = I - 2P$ is an orthogonal (and symmetric) matrix. In fact, if $P$ projects onto a subspace $U$ then $Q$ is the reflection through $U^{\perp}$.
\end{note}

\begin{definition}
Let $A$ be an $n \times m$ matrix with $\mathrm{rank}(A) = m$ and let $\bs{a}_1,\dots,\bs{a}_m$ be the columns of $A$. Apply the Gram-Schmidt algorithm to the columns and construct an orthonormal basis $\{ \bs{w}_1,\dots,\bs{w}_m \}$ of the column space. Project the columns onto the basis
\begin{align*}
\bs{a}_1 &= (\bs{w}_1 \cdot \bs{a}_1) \bs{w}_1 \\
\bs{a}_2 &= (\bs{w}_1 \cdot \bs{a}_2) \bs{w}_1 + (\bs{w}_2 \cdot \bs{a}_2) \bs{w}_2 \\
& \ \ \vdots \\
\bs{a}_m &= (\bs{w}_1 \cdot \bs{a}_m) \bs{w}_1 + (\bs{w}_2 \cdot \bs{a}_m) \bs{w}_2 + \cdots + (\bs{w}_m \cdot \bs{a}_m) \bs{w}_m
\end{align*}
where $\bs{a}_k \in \mathrm{span} \{ \bs{w}_1 , \dots , \bs{w}_k \}$ by construction. Write as matrix multiplication
$$
A = Q_1R_1
$$
where
$$
Q_1 = \begin{bmatrix} & & \\ \bs{w}_1 & \cdots & \bs{w}_m \\ & & \end{bmatrix}
\hspace{5mm}
R_1 = \begin{bmatrix}
\bs{w}_1 \cdot \bs{a}_1 & \bs{w}_1 \cdot \bs{a}_2 & \cdots & \bs{w}_1 \cdot \bs{a}_m \\
& \bs{w}_2 \cdot \bs{a}_2 & \cdots & \bs{w}_2 \cdot \bs{a}_m \\
& & \ddots & \vdots \\
& & & \bs{w}_m \cdot \bs{a}_m
\end{bmatrix}
$$
Extend the basis to an orthonormal basis $\{ \bs{w}_1 , \dots , \bs{w}_m , \bs{w}_{m+1} , \dots , \bs{w}_n \}$ of $\mathbb{R}^n$ where \\ $\{ \bs{w}_{m+1} , \dots , \bs{w}_n \}$ is {\it any} orthonormal basis of the orthogonal complement $\mathrm{col}(A)^{\perp}$ and let
$$
Q_2 = \begin{bmatrix} & & \\ \bs{w}_{m+1} & \cdots & \bs{w}_n \\ & & \end{bmatrix}
$$
Finally, the {\bf QR decomposition} of $A$ is
$$
A = QR =
\begin{bmatrix} Q_1 & Q_2 \end{bmatrix}
\begin{bmatrix} R_1 \\ 0 \end{bmatrix}
$$
where $Q$ is a $n \times n$ orthogonal matrix and $R$ is a $n \times m$ upper triangular matrix. See \href{https://en.wikipedia.org/wiki/QR_decomposition}{Wikipedia: QR decomposition} and also \cite[p.437]{KN}.
\end{definition}

\begin{example}
Compute the QR decomposition for the matrix
$$
A = \begin{bmatrix} 1 & 1 & 1 \\ 0 & 1 & 1 \\  1 & 1 & 0 \\  0 & 0 & 0 \\ \end{bmatrix}
$$
In a previous example, we found an orthonormal basis of the column space
$$
\bs{w}_1 = \frac{1}{\sqrt{2}} \begin{bmatrix} 1 \\ 0 \\ 1 \\ 0 \end{bmatrix}
\hspace{5mm}
\bs{w}_2 = \begin{bmatrix} 0 \\ 1 \\ 0 \\ 0 \end{bmatrix}
\hspace{5mm}
\bs{w}_3 = \frac{1}{\sqrt{2}} \left[ \begin{array}{r} 1 \\ 0 \\ -1 \\ 0 \end{array} \right]
$$
Extend to an orthonormal basis of $\mathbb{R}^4$ by $\bs{w}_4 = \begin{bmatrix} 0 & 0 & 0 & 1 \end{bmatrix}^T$. Therefore we have
$$
Q =
\begin{bmatrix}
1/\sqrt{2} & 0 & 1/\sqrt{2} & 0 \\
0 & 1 & 0 & 0 \\
1/\sqrt{2} & 0 & -1/\sqrt{2} & 0 \\
0 & 0 & 0 & 1
\end{bmatrix}
$$
and
$$
R =
\begin{bmatrix}
\bs{w}_1 \cdot \bs{a}_1 & \bs{w}_1 \cdot \bs{a}_2 & \bs{w}_1 \cdot \bs{a}_3 \\
0 & \bs{w}_2 \cdot \bs{a}_2 & \bs{w}_2 \cdot \bs{a}_3 \\
0 & 0 & \bs{w}_3 \cdot \bs{a}_3 \\
0 & 0 & 0
\end{bmatrix}
=
\begin{bmatrix}
\sqrt{2} & \sqrt{2} & 1/\sqrt{2} \\
0 & 1 & 1 \\
0 & 0 & 1/\sqrt{2} \\
0 & 0 & 0
\end{bmatrix}
$$
\end{example}

\begin{note}
The Gram-Schmidt algorithm shows that the QR decomposition exists but it is not the most efficient way to compute the QR decomposition. Software such as MATLAB {\tt qr} (see \href{https://www.mathworks.com/help/matlab/ref/qr.html}{documentation}) and SciPy {\tt scipy.linalg.qr} (see \href{https://docs.scipy.org/doc/scipy/reference/generated/scipy.linalg.qr.html}{documentation}) which is built on LAPACK (see \href{https://www.netlib.org/lapack/lug/node128.html#secorthog}{documentation}) use elementary reflectors to construct the matrices $Q$ and $R$.
\end{note}